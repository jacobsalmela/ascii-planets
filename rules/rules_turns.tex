% !TEX TS-program = XeLaTeX
\asciiplanets\ is played in turns.  With the exception of the solo mode (see \nameref{sec:playercount}, p.\pageref{sec:playercount}), the number of turns is not set beforehand.  Instead, the players take turns until one of several game end conditions (see \nameref{sec:endgame}, p.\pageref{sec:endgame}) is met.
\newline\newline
Each turn consists of five phases:
\begin{enumerate}
	\item \textbf{\dice}: The five dice are rolled.  Any player may do this (as the results will be the same for all players)
	\item \textbf{\diplomacy}: The players discuss the results and may collectively spend \currency\ to reroll some of the dice (see \nameref{sec:diplomacy}, p.\pageref{sec:diplomacy})
	\item \textbf{\disasters}: All dice still showing the value \texttt{1} are rerolled.  This may result in \disasters\ being triggered (see \nameref{sec:disasters}, p.\pageref{sec:disasters})
  \item \textbf{\development}: This is where most of the game action takes place. The die faces are now considered final and the players take actions simultaneously using the die results available to them.
  \item \textbf{\deployment}: Players may deploy \squadrons\ to perform one or more \military\ actions (see \nameref{sec:military}, p.\pageref{sec:military})
\end{enumerate}
The players then begin another turn if no game end conditions have been met.  Otherwise, the scores will be calculated after the final \deployment\ phase and the player with the most points wins.

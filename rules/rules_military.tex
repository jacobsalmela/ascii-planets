% !TEX TS-program = XeLaTeX
The art of war is of vital importance to your \planet.  
Mustering and deploying \squadrons\ of \starships\ allows you to attack opponents or even establish \battleships\ in their territory.  
There are also ruthless \pirates, a non-player enemy faction which you may wage war against.
\newline\newline
\military\ units are produced by creating \starships and recorded on a player's \military\ track with (\gainmilitarysymbol).
Each time one of your \starships\ is \activated, you may cross off an \astronaut\ from your \population\ track (\gainastronautsymbol\ becoming \useastronautsymbol).
When two boxes are filled in, a \squadron becomes available (the \emptysquadron\ to the right of the two boxes with \gainmilitarysymbol)
Checking the final box in the track triggers the end of the game and provides a scoring bonus even if you do not \textit{use} the squadron.
\newline\newline
If you want to deploy a \squadron\ in the \deployment\ phase, mark the \emptysquadron\ with an \gainsquadron to show it has been used.
Then, gain 1 point of \power\ (2 points instead, if you have researched \shields\ or \lasers).  
\newline\newline
As an attack is being resolved, players may deploy additional \squadrons.  
Only after no one wishes to deploy any more \squadrons\ are the effects resolved. 
You may deploy several \squadrons\ at once.
\newline\newline
\textbf{Example}: You want to have your \squadrons\ on reserve in the event of an attack.  
Over the course of several turns, you have gained 6 \military\ units.
The \military\ track will have 6 \gainmilitarysymbol's.  
The three \emptysquadron boxes remain unchecked.
When an attack arrives, you decide to deploy all three \squadrons by checking the \squadron box (an \emptysquadron\ becomes \gainsquadron)
\newline\newline
\power\ is to be used instantly for \military\ actions: it does not carry over to the next turn.
\newline\newline
You may use 1 point of \power\ to defend against attacks:
\begin{itemize}
  \item Cancel a \pirate\ raid on your \planet\ (see \nameref{sec:disasters}, p.\pageref{sec:disasters}).
  \item Reduce the \power\ of an opponent's attack on you by \texttt{1}.
  \item Reduce the \power\ of an opponent's attack on another player by \texttt{1}, provided you have \reach\ to the defender.
  \item Prevent the destruction of your \starship\ by the \terrorism\ \disaster\ (see \nameref{sec:disasters}, p.\pageref{sec:disasters}).
\end{itemize}
You may use \texttt{1} point of \power\ to attack:
\begin{itemize}
  \item Destroy a \pirate\ ship, provided there are unraided \pirates\ on a sheet you have \reach\ to (see \nameref{sec:reach}, p.\pageref{sec:reach}).  This yields you three \currency\ (\gaincurrency\gaincurrency\gaincurrency).  When all \pirates\ on your sheet are destroyed (regardless of who destroyed them), you immediately gain \texttt{2} \happiness.
  \item Attack another player's \planet\ to whom you have \reach.  They lose \texttt{1} \currency\ (if able) and gain \texttt{1} \unhappiness. If they lost \currency\, you gain \texttt{1} \currency.
\end{itemize}
You may use \texttt{2} points of \power\ to:
\begin{itemize}
  \item Establish a \battleship\ on another player's territory to which you have \reach.  There is no limit to the amount of \battleships\ you can deploy.
  \item Raze an opponent's \battleship\ on your territory.  The \battleship\ is crossed over and cannot be activated.  \battleships\ are permanent figures on other players' territory.  They can only be established by a successful attack with \power\ \texttt{2} or more--not through regular construction.
  \item A \battleship\ is established by drawing a square on any free area on the opponent's map, just as you would draw an \armament\ on your own map.  Mark it with your \insignia/initials/distinguishing mark.
  \item The \battleship\ may not touch or overlap other \armaments.  If its outline intersects Nebulus terrain, pay \texttt{1} \currency.
  \item The opponent may not build in the area taken by another player's \battleship.
\end{itemize}
\textbf{Example}: Kashykk has three \squadrons\ and Alderaan has one.  
The Kashykk player deploys two \squadrons\ for \texttt{2} \power\ to establish an \battleship\ in Alderaan.  
The Alderaan player deploys their only \squadron, reducing Kashykk's \power\ to \texttt{1}.  
The Kashykk player now has a choice: they may use the single point for a regular attack or establish the \battleship\ by deploying their remaining \squadron\ for a total of \texttt{2} unopposed \power.
\newline\newline
\battleships\ can be \activated\ with a die showing \texttt{6}. 
When \activating\ an \battleship, you can \activate\ any one \armament\ type, counting both your own \armament\ and the colonized players' \armaments\ (as if they belonged to you).  
Those players then gain one \unhappiness.

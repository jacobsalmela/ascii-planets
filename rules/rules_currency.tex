% !TEX TS-program = XeLaTeX
\currency\ is produced through \factories\ but can also be gained by advancing \trade, \tech, and \culture.  Available \currency\ is recorded on the \currency\ track by filling the squares with the capital letter (\gaincurrency).  When spent, the capital \gaincurrency' have a line drawn through them, becoming the dollar sign (\spendcurrency).
There are many uses for \currency:
\begin{itemize}
  \item Spending one \currency\ (\spendcurrency) lets you modify a die value by \texttt{1} in either direction in the \development\ phase for the purpose of \construction, \activation, \trade, or \culture.  The same die can be modified several times by spending more \currency.  The die is not turned – it retains its original value for the other players.  A modified die value may exceed \texttt{6} (rising to \texttt{7+} – it does not "go over" to \texttt{1}) but cannot go lower than \texttt{2}.
  \item One \currency\ (\spendcurrency) lets you also hire a new \astronaut.  (Add \gainastronautsymbol\ to \population\ track.)
  \item With three \currency\ (\spendcurrency\spendcurrency\spendcurrency), you may check one box of a \tech, \culture, or \military\ track (reflecting patronage of scientists and artists, or hiring \pilots, respectively).
  \item Five \currency\ (\spendcurrency\spendcurrency\spendcurrency\spendcurrency\spendcurrency) lets you provide advanced \tech\ and clean energy to the people, increasing \happiness\ (see \nameref{sec:happiness}, p.\pageref{sec:happiness}) by one.
  \item \currency\ can be spent for rerolls in the \diplomacy\ phase (see \nameref{sec:diplomacy}, p.\pageref{sec:diplomacy}).
\end{itemize}
Filling the entire \currency\ track triggers the end of the game and provides a scoring bonus.

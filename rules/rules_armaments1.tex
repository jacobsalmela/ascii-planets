% !TEX TS-program = XeLaTeX
Construction is one of the main ways for the \planet's inhabitants to expand into \outerspace.
\armaments\ are constructed by connecting four dots on the map with lines to make a square.
\newline\newline
You can create the following types of \armaments, which when staffed and \activated, produce an advantage for your planet (all \armaments\ earn you points):
\begin{itemize}
  \item Academy (Allows you to train an \astronaut\ by adding a \gainastronautsymbol\ to your \population\ track)
  \item Factory (Allows you to gain one \currency by adding a \gaincurrency\ to your \currency\ table)
  \item Lab (Allows you to cross off a box on the \tech\ track)
  \item Starship (Allows you to train \astronauts\ to form \squadrons, which can be used for military actions)
  \item Battleship (Allows you to activate any other \armaments\ on your's or another player's map, which you have \reach)
  \item Space Station (provides a considerable point bonus at the end of the game)
\end{itemize}
%TODO add example of drawing building/ship on the map
The number of \armaments\ you may construct on a turn is equal to the combined value of dice used for construction.  For example, the \academy\ and \factory\ each cost \texttt{8}; if you rolled a total of \texttt{16}, you could create two of either \armaments\ or one of each.  You may use one or more \greatperson\ bonuses from the \population\ track (see \nameref{sec:population}, p.\pageref{sec:population} to increase the amount you have to spend by \texttt{4}).

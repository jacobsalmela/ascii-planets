% !TEX TS-program = XeLaTeX
Example: The 3 \whitedice\ display values \texttt{2}, \texttt{3} and \texttt{4} and you have \texttt{1} \greatperson\ available.  If all three dice are used for construction, the total number possible is \texttt{2+3+4 = 9}.  This is enough to construct an \academy\ or \factory\ building, each of which cost 8 to complete.  The remainder (\texttt{9-8 = 1}) is lost, because no \fortifications\ cost \texttt{1}.  Alternatively, you could use only the \texttt{4} result and the \greatperson\ bonus of \texttt{+4}, sparing the \texttt{2} and \texttt{3} results for \activating\ \fortifications, \trade, or \culture.
\newline\newline
There are six different \fortification\ types (see \nameref{sec:fortifications}, p.\pageref{sec:fortifications}) divided into three \levels.
\begin{itemize}
	\item To construct a \level\ II building, you need to have at least one staffed \fortification\ of each \level\ I type. Correspondingly, \level\ II \fortifications\ are a prerequisite for \level\ III \fortifications.
	\item Constructing all \level\ III \fortifications\ triggers the end of the game and provides a scoring bonus.  Some rules always apply when constructing:
  \begin{itemize}
    \item Any number of \fortifications can be constructed on a turn
    \item A \fortification\ has to be completed (its cost fully paid for) on a single turn. (You may not start constructing a \fortification\ and finish it on a later turn.)
    \item \fortifications\ can touch but not overlap, including \fortifications\ destroyed by game effects
    \item If any part of a building’s outline would intersect Mountainous terrain or a Nebula, the construction costs one (\spendcurrency) \currency. (If you have no \currency, you may not build there.)
    \item A \fortification\ may be staffed (with a letter inside) or empty.  The \spacestation\ needs no staff.
    \item Once complete, a \fortification\ may be staffed by crossing one of your available \astronauts\ (\gainastronautsymbol\ becoming \useastronautsymbol) and marking the letter (e.g. \academysymbol\ for \academy) inside the \fortification. The \astronaut\ is now a \specialist\ working in that \fortification.
    \item You must staff \fortifications\ on the same turn they were constructed, if able. (You cannot carry empty \fortifications\ and idle \astronauts\ from one turn to the next).  Staffed \fortifications\ can be \activated\ using a die.  When \activated, each \fortification\ produces a distinct advantage (see \nameref{sec:fortifications}, p.\pageref{sec:fortifications}).
  \end{itemize}
  \item A single die may activate all of your \fortifications\ of the same type.
  \item A single die may only activate one type of \fortification.
  \item A \fortification\ may be activated any number of times each turn.
  \item A \fortification\ may be activated on the same turn it was built, provided it is staffed and a suitable die is still available.
\end{itemize}
To activate a \fortification, the die must have a high enough value.  An \academy\ can be activated with any die equal to or greater than \texttt{2}, while \starships\ can only be activated with a \texttt{5} or \texttt{6}.  The die values necessary to activate a \fortification\ are shown in the table on the \planetsheet.
\newline\newline
A \texttt{6} may activate a \spacestation\ or any other \fortification\ type. A \texttt{3} can activate \factory\ or \academy, while a \texttt{2} may only activate \academy.
\newline\newline
Example: You have two \labs\ and an \academy\ and access to dice showing the results \texttt{2}, \texttt{4} and \texttt{5}.  You may use both \texttt{4} and \texttt{5} to activate the two \labs\ (for a total output of \texttt{4} \tech) and the \texttt{2} to activate the \academy\ (to produce one \astronaut\ for the \population\ track).
% TODO: Convert this to a table
% A new \astronaut\ (\gainastronautsymbol) is added to the \population\ track
% New \currency\ (\gaincurrency) is added to the \currency\ track
% The leftmost box of the \tech\ track (or any one of its two branches) is checked.
% An \astronaut\ is crossed off, becoming a \pilot\ (the leftmost box of the \military\ track is checked) N/A (The \spacestation\ is never activated, nor does it need a \specialist: it is the hub where you operate your inter-planetary actions.  You may only build one.)
% \newline\newline
% \battleship: You may activate one building type as if affected opponents’ \fortifications\ also belonged to you. The opponents gain 1 \unhappiness.

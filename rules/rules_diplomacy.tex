% !TEX TS-program = XeLaTeX
The \diplomacy\ phase comes right after the dice are first rolled at the beginning of a turn.  The players have a chance to reroll any or all dice.  They might, for example, choose to reroll all dice displaying a value of 1 (and thus, about to trigger \disasters\ in the following step).  A reroll is only possible if the players collectively spend \currency\ equal to the number of players (e.g. \spendcurrency\spendcurrency\spendcurrency\spendcurrency\ in a 4-player game).
\begin{itemize}
  \item The requirement can be met by each player paying one \currency, but this need not necessarily be the case.  The players must negotiate.  Even one player alone may perform a reroll by paying the cost.  The other players need not agree; they only need not pay.
  \item The cost is paid only once per reroll, no matter the number of dice rerolled.
  \item Any number of dice can be rerolled any number of times.
  \item If there are several different proposals for rerolls, follow the Order of Play rules to determine who goes first (see \nameref{sec:order}, p.\pageref{sec:order}).
\end{itemize}

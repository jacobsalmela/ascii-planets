% !TEX TS-program = XeLaTeX
In multiplayer games, it is relevant to consider which players have \reach\ to 
one another for the purposes of \military\ actions or \trade.  This is 
determined by two things:
\begin{itemize}
	\item Seating order: you will have \reach\ to the players sitting next to you 
        after you build and staff at least one \starship. (exception: 2-player 
        games, see \nameref{sec:playercount}, p.\pageref{sec:playercount}).
	\item Progress on the \tech\ track: once a player has researched \warpdrive\ (
        and they have at least one ship), their \reach\ will expand and allow 
        travel to any other player's \planet.
\end{itemize}
\reach\ is most easily thought of as the ability to leave your \planet\ and 
venture into the \outerspace\ in order to reach other players' \planets.  At the
beginning of the game, your \convoys\ and \squadrons\ cannot travel in 
\outerspace\ until you build at least one \starship.  Once this is 
accomplished, they can travel to the \planet\ on your immediate right or left; 
however, they cannot cross the vastness of \outerspace\ that divides the other
boards, nor can they travel to \reach\ the players on the opposite side of the 
table (more than one planet away) until \warpdrive\ is unlocked.
\newline\newline
Thus, once you have at least one \starship\ you have \reach\ to your neighbour's
\planet\ to your immediate left and right.  Once your \planet\ has researched 
\warpdrive\ (see \nameref{sec:technology}, p.\pageref{sec:technology}), you can 
travel to any \planet\ of your choosing, and therefore have \reach\ to all 
players' \planets.
